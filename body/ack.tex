%%==================================================
%% thanks.tex for SJTU Master Thesis
%% based on CASthesis
%% modified by wei.jianwen@gmail.com
%% version: 0.3a
%% Encoding: UTF-8
%% last update: Dec 5th, 2010
%%==================================================

\begin{thanks}

在论文定稿之际,心中颇多感慨。论文的写作过程是艰苦的,但我有幸得到了各位师、领导、同学、朋友、同事和亲人的教诲和帮助。没有他们,也就没有论文的最终成果。

首先我要特别感谢我的导师黄林鹏教授。本文是在黄老师的悉心指导下完成的。从本文的选题、构思、写作、修改直到最后定稿,都凝聚着导师的智慧、才华与心血。黄老师学识渊博、治学严谨求实、看待问题高屋建瓴、对待工作非常负责,为人随和坦诚,他的言传身教将使我终生受益。师恩难忘,在此,向黄老师表达我最诚挚的敬意与谢意!我还要感谢黄老师的博士生李素敏,每周都坚持和我讨论毕设的进展和阶段性成果,讨论算法的可行性和可实现性,并运用了多年写论文的经验帮我修改论文和提出意见,在此表示真诚的感谢。同时我还要感谢电子信息与电气工程学院的很多老师。从我入学至今,他们在有形无形中、有意无意中给予我很多知识,给我的选题、论证提供了诸多的启发与帮助。在此向这些老师以及给予我帮助的其他老师表示诚挚的感谢。我还要感谢我的同窗同学,他们在论文写作过程中给予了我很多的帮助。

最后我还要感谢我的家人,感谢他们给予我生活和精神上的关心、支持和鼓励,才能毫无后顾之忧的学习。

值此论文完成之际,我谨向以上曾经给予我指导和关心的老师、同学、同事和家人意最诚挚的谢意!

\end{thanks}
